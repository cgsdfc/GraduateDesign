本章从模型,指标和数据集三个方面回顾了本领域的研究现状。
在模型方面,基于Seq2Seq的对话系统有多种优势,例如:
能够直接生成响应,
能端对端的训练,
能从大量语料数据中自动学习语言规律,减少写死的规则,
能利用话题和情感等多种表征,等等。
然而这类系统也有弊端,
比如无法准确控制生成的内容,
偏向于生成单调的响应,
没有人格一致性等等。

在指标方面,对话系统的指标还不能像机器翻译的指标那样准确的评价系统的性能。
% -- this is totally disscusion.
目前最大的问题可能是如何提高指标和人类评价的相关性。
导致这个困境的核心原因可能是对话的响应具有固有的语义多样性,
这个事实可能导致两个结果:1. 用机器翻译的表征,比如单词层面的n-gram,字符层面的n-gram,
对齐等等,无法有效捕捉语义层面的信息。
2. 单纯使用衡量相似性的指标无法捕捉多样性的维度,从而与人类评价不符。

在数据集方面,本领域已经积累了来自不同领域的数据,例如技术性问答,社交媒体上的闲聊,电影对白和字幕,等等。
一些数据集在领域上出现了重合,
比如电影对话数据集(Movie Dialogs Corpus),
电影三元组数据集(MovieTriples)和Movie-DiC的领域都是电影对白;
OpenSubtitles和SubTle的领域都是电影字幕;
Twitter数据集(Twitter Corpus),
Twitter对话数据集和Twitter三元组数据集的领域都是Twitter短文本闲聊。
% -- This is totally future_work.
尽管人类标注是对话文本的重要补充,但是由于其代价的高昂,
所得的数据量往往比较少;
另一方面,元信息是一类在数据采集过程中伴随对话文本同时获得的描述性信息,
比如对话者的性别(Gender)和角色(Character)等等。
相比与人类标注,元信息更容易获取,规模不受限制。
并且,人类标注和元信息都有助于模型生成更加多样和连贯的响应
\upcite{persona,Topic_Aware,ECM}。
因此,数据集的收集过程应该充分保留原始数据中的元信息,
对话系统应该充分利用元信息。
同时,还应该发展从对话文本中无监督的提取元信息的方法。
