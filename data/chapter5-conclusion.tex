\chapter*{结论}\label{ch:conclusion}
\addcontentsline{toc}{chapter}{结论}

% -------------------------- %
% -------- Summary --------- %
% -------------------------- %
\section*{总结}\label{sec:conclusion}
% -- Related Work -- %
本文对生成式对话领域的模型,数据集和指标进行了一次深入考察。
本文首先介绍了本领域的研究现状,
接着介绍了生成式模型的定义,RNN语言模型和Seq2Seq框架。
本文着重介绍了本领域的常见指标,包括基于词重叠的指标,基于词嵌入的指标,
衡量概率语言模型性能的困惑度,以及专门为生成式对话设计的ADEM和RUBER等等。
最后,本文介绍了本领域常用的开放领域对话数据集。

% -- Methodology -- %
在众多模型,数据集和指标中,本文选择了Serban等人在文献\cite{VHRED}
中使用的三个模型HRED,LSTM和VHRED。
在数据集方面,本文选择了公开的,代表了不同领域的三个数据集,
分别是Ubuntu对话数据集,OpenSubtitles和LSDSCC数据集。
在指标方面,本文尽可能涵盖了对话领域使用过或者提出的指标。
本文的主要工作是:
\begin{enumerate}
    \item 在多个数据集上训练多个模型。
    \item 在训练结果上运行多个评价指标。
    \item 对指标进行多种数据分析。
\end{enumerate}
实验配置与文献\cite{HowNot}大致对齐。
实验数据包括多个模型和数据集的组合在不同指标上的得分,以及这些模型输出的响应。
由于时间关系,本文只分析了得分的数据,把分析模型的输出留给后续工作。

% -- Experiment -- %
本文分析了系统层面得分和句子层面得分。
系统层面得分是是对模型在一个数据集上的表现的粗粒度考察,
它可能掩盖了一些事实,
但是便于综合考察模型、数据集和指标三者的整体关系。
参照文献\cite{HowNot},本文对指标进行分组,并围绕各组指标进行分析。
各模型的系统层面得分随着数据集的变化和指标的变化有较大差异,
比较稳定的是同一个数据集上各模型的排名,
一般HRED和VHRED比LSTM得分高。
模型在Ubuntu对话数据集上的各项指标通常更好。
除了实验过程本身的噪音外,
主要的原因是开放领域对话数据集的特征变化大,
模型无法将一个数据集上的性能完全迁移到另一个数据集上。

本文还考察了句子层面得分的单变量分布,
从分布图像中发现了指标的集群现象。
集群现象反映了提取相同特征的指标有着相似的分布。
指标分布的图像大致可分为两类:
\begin{enumerate}
    \item 基于词嵌入的指标的图像是钟型曲线,接近高斯分布。
    \item 基于词重叠的指标的图像是不对称的双峰曲线,大量句子集中在均值附近很小的范围。
\end{enumerate}
本文尚不清楚指标的具体算法或者模型是如何影响分布的,有待后续研究。

% -------------------------- %
% ------ Future Work ------- %
% -------------------------- %
\section*{展望}\label{sec:future_work}
在实验结论的基础上,
本文对生成式对话领域的模型,数据集以及指标提出几点展望。

% -- Model -- %
在模型方面,尽管Seq2Seq框架在机器翻译领域取得成功,
但是在更加困难的对话响应生成领域,它的表达力遇到了瓶颈。
因此,本文建议尝试新的模型体系结构。
Transformer框架\upcite{Transformer}是一种在大规模语料库上预训练的,
能根据特定任务微调的语言模型。
基于Transformer框架的模型在多项自然语言处理任务中取得了目前最高水平,
将其应用到生成式对话领域有望提升模型的性能。
其他值得尝试的体系结构有对抗生成网络和强化学习,
Li等人在这方面获得了意义重大的进展\upcite{deep_RL,Adversarial}。
另一方面,本文建议在模型中添加额外的特征,比如感情色彩\upcite{ECM},
主题词\upcite{Topic_Aware}和对话者身份信息\upcite{persona}等等。
添加额外特征能使模型的输出在这些特征上具有一致性,从而提高人类评价的得分。

% -- Dataset -- %
在数据集方面,通过在线聊天收集的对话样本具有很高的多样性,
这和人类在这种开放和匿名的平台的表现有密切联系。
从概率的角度,数据集的样本分布可以看作是非常多个随机变量的叠加,
如果这些随机变量都是独立同分布的话,
整个数据集的样本分布就趋向于高斯分布。
面对复杂的数据集,本文建议使用概率统计工具分析它在
多个方面的分布情况,例如情感分布,对话轮数分布等等。
了解数据集的统计特征有助于理解在这个数据集上训练模型的难度。

% -- Metric -- %
在指标方面,本文的实验结论补充了文献\cite{HowNot}的结论。
为了深入理解对话生成问题,本文提出一个问题:什么样的对话才是好的对话?
这是一个重要的问题,它不仅指导着指标的构建,还决定了模型优化的方向。
人类的对话是在自然环境和社会环境中进化出的一种语言现象,
它根据不同场合和参与者的变化而变化的,非常复杂。
所以这个问题可能难以回答,甚至没有普适性的答案,

从实用的角度,
可以通过实验发现在某些数据集上好的对话应有的特征。
直观的来说,
在Ubuntu对话数据集上,对话以“提问-解决问题”为主,
好的对话应该能帮助人们解决问题,
因此应该和具体的技术话题有较高的相关性;
在Twitter对话数据集上,
对话主要是人们发布的个人状态信息,经常带有感情色彩,
而且人们期望从响应中看到相同的话题或者新奇的事物,
因此好的对话应该考虑情感因素,
关注主题的同时又具有一定的多样性,
在LSDSCC数据集上,对话主要是人们发表对电影的点评,
人们一般希望和看过同类电影的人一起讨论,
虽然有时候有人会发表极端的评价,
但是从长远来看,人们不希望总是看到极端的评价,
所以好的对话应该是对相关电影的中肯评价,
并且带有个性化看法。

综上所述,不妨给“什么样的对话才是好的对话”加上“在某个数据集上”的限定词,
从某个数据集中发现好的对话的模式,
然后设计出能捕获这些模式的指标,
并用它评价在该数据集上训练的模型。
具体来说,本文提出把“设计出和人类评价相关度高的指标”这一任务分解为若干个小任务:
\begin{enumerate}
    \item 把问题限制在某个数据集上。
    \item 找出这个数据集上人类评价高的对话所具有的特征。
    \item 设计出能准确捕获这些特征的指标。
    \item 用人类评价验证指标在对应数据集上的有效性。
\end{enumerate}

本文把这个方法称为数据驱动的指标构建法(Data-Driven Metric Construction)。
必须指出,第二步和第三步具有很大的难度。
第二步一般需要人类评价员对数据集的样本打分,
这将导致带有人类评价的样本比较稀少,
要求指标的泛化能力比较强\upcite{ADEM}。
第三步涉及的特征比较抽象,要求指标的建模能力比较强。
