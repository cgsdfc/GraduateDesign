% !Mode:: "TeX:UTF-8"
% Author: Zhengxi Tian
% Email: zhengxi.tian@hotmail.com

% 中英文摘要
\begin{cabstract}
    尽管基于序列到序列的生成式对话系统已经能够生成自然而流畅的响应,
    这类模型普遍存在着生成单调响应(Generic Response)的倾向。
    对话系统的目标是生成多样的,有意义的,能引起人们兴趣的对话。
    为了实现这一目标,学者们提出了各种模型,
    从不同角度解决单调响应的问题。
    但是由于缺乏好的自动化评价指标,模型的评估高度依赖于人类评价,
    导致评估系统的代价高昂,规模难以扩大。
    为了了解不同的评价指标的优缺点,
    本文在三个公开数据集上训练了三个生成式对话模型,
    测定不同指标的系统层面得分和句子层面得分。
    我们发现,不同数据集上的模型在不同指标上的得分没有完全的一致性。
    总体来说,数据集对得分的影响大于模型对得分的影响,
    使用相同特征的指标具有相似的句子层面得分分布。
    本文从经验上总结了上述现象的原因,包括
    合理的响应空间过于庞大,给评价增加了难度;
    模型在不同数据集上的泛化能力有待加强;
    指标的分布各异,给评价造成了混乱。
    本文为解决上述问题提出了若干方向,
    包括使用新的模型体系结构,
    研究开放式领域对话数据集的统计规律
    和发展特定数据集上的指标等等。
\end{cabstract}

\begin{eabstract}
    Although the Seq2Seq-based generative dialogue systems are
    able to generate natural and
    fluent responses,
    they have been widely known for preferring to generate
    simple and repeated responses.
    Towards the goal of generating diverse, meaningful and
    engaging dialogues,
    many researchers proposed various methods to address the
    problem of low-quality responses.
    However, it is known that the
    lack of good automatic evaluation metrics
    has led field to rely heavily on human evaluation,
    which is both expensive and unscalable.
    To better understand the pros and cons of various
    evaluation metrics,
    we trained three generative models on
    three public-available datasets and measured their
    performances with various metrics on system level
    as well as utterance level.
    We found that there is no universal consistency among
    the scores of all models on all datasets with all metrics.
    However, the datasets generally pose a heavier impact on
    the scores than the models do.
    In addition, the distributions of
    utterance-level scores are similar if the metrics use the same
    set of features.
    We made empirical explanation on the
    observation that the enormous space for reasonable responses
    makes the evaluation harder to tackle.
    Meanwhile, the ability of the models to generalize across
    various datasets remain highly enhanceable.
    Worse still, the highly diversed distributions of scores
    of various metrics makes the results somehow confusing.
    Based on the empirical results, we pointed out several
    directions for future work, including using new model
    architectures, investigating the statistical nature of
    the open-domain dialogue datasets and
    developing dataset-specific metrics.
\end{eabstract}
