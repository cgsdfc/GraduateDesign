% !Mode:: "TeX:UTF-8"
% Author: Zhengxi Tian
% Email: zhengxi.tian@hotmail.com

\chapter{绪论}\label{ch:绪论}

\section{课题研究背景}
会话代理(Conversational Agent),又称聊天机器人(Chatbot),是一种能和人类用自然语言交流的系统,表现在它能够理解人类的问题的意义,并给出合适的回答。聊天机器人的出现和流行对现代人的生活和人工智能的研究都具有重大的意义。对于生活在钢筋水泥森林的现代人来说,倾诉心事可能是最难满足的需求,而微软小冰的出现,给他们带来一个可以倾诉的聊天伴侣,有利于缓解现代人广泛存在的心理问题。虚拟助手的出现,让原本冰冷的电子产品变得人格化,极大的改善了人机交互的接口。对着手机说一声“Siri,我要听歌”,便有一个温柔的,善解人意的声音和你交互;对着笔记本电脑说一声“小娜,帮我查查明天的天气”,小娜就娓娓道来。而这些场景都是会话代理在现代生活的应用。而对于人工智能研究来说,能像人类一样交流本来就是图灵对人工智能下的定义:著名的图灵测试。随着研究的发展,图灵测试作为一种测评手段可能会过时,但是能和人类自然的交流却始终是人工智能的一个重要目标。

实现聊天机器人的关键技术就是对话系统(Dialogue System),它是一种能理解自然语言输入并产生恰当的响应(Response)的系统。互联网和社交媒体的普及,创造了大量的对话数据,为构建数据驱动的对话系统提供了可能\upcite{Ritter:2011:DRG:2145432.2145500}。而深度学习与自然语言处理的结合,产生了序列到序列(Seq2Seq)模型,为对话系统充分利用大数据奠定了基础。在这些有利条件的促进下,对话系统正处在快速发展中。为了加速模型的开发,我们需要自动化的评价模型的性能,然而这恰好是一个公开的难题,尤其是对于聊天机器人这类开放领域的对话系统[7]。因此,研究对话系统的评价指标便成为一个具有较高学术价值和现实意义的课题。

\section{课题研究意义}


\section{课题研究内容}


\section{论文组织结构}
本论文主要研究了深度学习,尤其是注意力机制和循环神经网络在情感分析中的应用。
考虑到情感分析领域中传统模型的浅层缺陷,和现在被广泛使用的神经网络的优缺点和性质。
本文提出了一个基于注意力机制的双向循环门控制单元框架,并且在四个标准数据集上进行了实验和量化分析。

本文的组织结构如下:

第一章:绪论。在绪论章节中,本文首先阐述了本课题的研究背景,并大致介绍了情感分析的研究现状和发展趋势等方面。
接下来,本文介绍了研究意义。其次,本文重点阐明了课题的研究内容和创新点。

第二章:相关知识和工作介绍。在做研究期间,本文的研究者查阅了大量的国内外相关文献,对情感分析中涉及到的知识做了详尽的说明。
包含计算语言学的普适方法极其局限性,机器学习和深度学习方法在情感分析中的相关应用。
并着重介绍了基于循环神经网络的语言模型,双向循环神经网络,门控制循环单元和注意力机制。

第三章:基于注意力机制的双向循环门控制单元框架。该章是本文研究的核心内容,也是本文的创新所在。
本章首先介绍了框架的详细结构,接下来,分别详细介绍了框架的主要三部分:“句子序列建模”的预注意力双向循环GRU,
“词特征捕捉”的注意力层设计,类解码器的后注意力单向GRU层。

第四章:实验结果与讨论。本章主要介绍了框架在四个标准数据集上的实验效果。首先介绍了公开的标准数据集极其组成。
然后详细阐述了评价方法,模型训练过程,超参数调节和实验结果等。最后从各个角度,例如对词嵌入进行可视化等,全面对实验进行了分析和讨论。

第五章:总结与展望。该章节对论文进行了总结,并指出工作中未来的研究方向。
