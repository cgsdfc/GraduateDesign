% !Mode:: "TeX:UTF-8"
% Author: Zhengxi Tian
% Email: zhengxi.tian@hotmail.com

% 学院中英文名,中文不需要“学院”二字
% 院系英文名可从以下导航页面进入各个学院的主页查看
% http://www.buaa.edu.cn/xyykc/index.htm
\school
{计算机}{School of Computer Science and Engineering}

% 专业中英文名
\major
{计算机科学与技术}{Computer Science and Technology}

% 论文中英文标题
\thesistitle
{基于注意力机制和双向循环门控制单元的\\情感分析研究}
{}
{Attention Aware Bi-directional Gated Recurrent Unit Based Framework for Sentiment Analysis}
{}

% 作者中英文名
\thesisauthor
{中文名字}{Firstname Lastname}

% 导师中英文名
\teacher
{导师名字}{Firstname Lastname}
% 副导师中英文名
% 注:慎用‘副导师’,见北航研究生毕业论文规范
%\subteacher{副导师}{subteacher}

% 中图分类号,可在 http://www.ztflh.com/ 查询
\category{TP312}

% 本科生为毕设开始时间;研究生为学习开始时间
\thesisbegin{2017}{10}{23}

% 本科生为毕设结束时间;研究生为学习结束时间
\thesisend{2018}{06}{11}

% 毕设答辩时间
\defense{2018}{06}{11}

% 中文摘要关键字
\ckeyword{自然语言处理,深度学习,情感分析,注意力机制,双向循环门控制单元}

% 英文摘要关键字
\ekeyword{Natural Language Processing, Deep Learning, Sentiment Analysis, Attention, Bidirectional GRU, }
