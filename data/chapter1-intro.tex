% !Mode:: "TeX:UTF-8"
% Author: Zhengxi Tian
% Email: zhengxi.tian@hotmail.com

\chapter{绪论}\label{ch:绪论}

\section{课题研究背景}
会话代理(Conversational Agent),又称聊天机器人(Chatbot),是一种能和人类用自然语言交流的系统,表现在它能够理解人类的问题的意义,并给出合适的回答。聊天机器人的出现和流行对现代人的生活和人工智能的研究都具有重大的意义。对于生活在钢筋水泥森林的现代人来说,倾诉心事可能是最难满足的需求,而微软小冰的出现,给他们带来一个可以倾诉的聊天伴侣,有利于缓解现代人广泛存在的心理问题。虚拟助手的出现,让原本冰冷的电子产品变得人格化,极大的改善了人机交互的接口。对着手机说一声“Siri,我要听歌”,便有一个温柔的,善解人意的声音和你交互;对着笔记本电脑说一声“小娜,帮我查查明天的天气”,小娜就娓娓道来。而这些场景都是会话代理在现代生活的应用。而对于人工智能研究来说,能像人类一样交流本来就是图灵对人工智能下的定义:著名的图灵测试。随着研究的发展,图灵测试作为一种测评手段可能会过时,但是能和人类自然的交流却始终是人工智能的一个重要目标。

实现聊天机器人的关键技术就是对话系统(Dialogue System),它是一种能理解自然语言输入并产生恰当的响应(Response)的系统。互联网和社交媒体的普及,创造了大量的对话数据,为构建数据驱动的对话系统提供了可能\upcite{Ritter:2011:DRG:2145432.2145500}。而深度学习与自然语言处理的结合,产生了序列到序列(Seq2Seq)模型,为对话系统充分利用大数据奠定了基础。在这些有利条件的促进下,对话系统正处在快速发展中。为了加速模型的开发,我们需要自动化的评价模型的性能,然而这恰好是一个公开的难题,尤其是对于聊天机器人这类开放领域的对话系统[7]。因此,研究对话系统的评价指标便成为一个具有较高学术价值和现实意义的课题。

\section{课题研究意义}
hello

\section{课题研究内容}


\section{论文组织结构}

