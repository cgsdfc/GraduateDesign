% !Mode:: "TeX:UTF-8"
% 任务书中的信息
%% 原始资料及设计要求
\assignReq
{采用深度学习的方法设计并实现能够对文本情感极性进行分析和分类的框架}
{设计并实现基于注意力机制和双向循环门控制单元的框架}
{设计并实现预注意力双向循环门控制单元层,注意力层和后注意力单向}
{门控制单元层   采用四个标准公开数据集以及GloVe词向量进行实验}
{阅读自然语言处理尤其是情感分析领域的相关论文}
%% 工作内容
\assignWork
{熟悉自然语言处理尤其是情感分析领域的相关研究}
{设计并实现基于注意力机制和双向循环门控制单元的框架}
{设计并实现预注意力双向循环门控制单元层,注意力层和后注意力单向}
{门控制单元层    对四个标准公开的数据集进行预处理和词向量转换,用于实验}
{总结并分析实验结果,可视化调参过程和词嵌入训练结果}
{撰写会议论文和毕业论文}
%% 参考文献
\assignRef
{Wang Y., Huang M., Zhu X., et al. Attention-based LSTM for Aspect-level}
{Sentiment Classification[A]\ \  Bahdanau D.,Cho K.,Bengio Y. Neural}
{Machine Translation by Jointly Learning to Align and Translate[J]}
{Goodfellow I., Bengio Y., Courville A., et al. Deep learning[M]}
{Mousa A. E., Schuller B. W. Contextual Bidirectional Long Short-Term Memory}
{Recurrent Neural Network Language Models: A Generative Approach to}
{Sentiment Analysis[A]\ \ Liu J., Rong W., Tian C., et al. Attention Aware}
{Semi-supervised Framework for Sentiment Analysis[A]}