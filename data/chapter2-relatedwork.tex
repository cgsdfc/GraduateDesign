% !Mode:: "TeX:UTF-8"
% Author: Zhengxi Tian
% Email: zhengxi.tian@hotmail.com

\chapter{相关工作}\label{ch:related_work}
\section{生成式模型}\label{sec:generative_model}
一个生成式模型定义了给定输入序列$X = x_1, x_2, \cdots, x_n$,
任意输出序列$Y = y_1, y_2, \cdots, y_m$的条件概率:
\begin{align}
    p(Y|X) = P(y_1, y_2, \cdots, y_m|x_1, x_2, \cdots, x_n)
    \label{eqn:generative_conditional_probability}
\end{align}
模型的训练目标就是在数据集$S$上最大化给定$X$,$Y$的对数概率(Log Probability):
\begin{align}
    \mathit{L} = \frac{1}{|S|} \sum_{(Y, X) \in S} \log p(Y|X)
\end{align}
从这个角度来看,语言模型\upcite{
DBLP:journals/jmlr/BengioDVJ03,
DBLP:conf/interspeech/MikolovKBCK10}(Language Model)和编解码器(Encoder-Decoder)都属于生成式模型,
因为它们都定义了条件概率$p(Y|X)$。
生成式模型把一个长度可变的序列$X$映射到另一个长度可变的序列$Y$,且$X$和$Y$的长度可以不相等。
循环神经网络(RNN)为这个问题提供了自然的解决方案。
% RNN
RNN的基本思想是:序列由有序的元素组成,每一个时刻(Time Step)输入并输出一个元素,同时更新内部的隐层状态(Hidden State)。
在时间轴上展开的RNN和一般的前馈神经网络(Feed Forward Neural Networks)很像,不过每一个时刻的权重矩阵都是共享的。
这个共享的权重矩阵$A$又被称为循环矩阵(Recurrent Matrix)。
循环矩阵的作用是保存输入序列的顺序信息,并把当前时刻的输入序列编码成一个定长向量。
图~\ref{fig:RNN_unrolled}\footnote{http://colah.github.io/posts/2015-08-Understanding-LSTMs/}
描绘了简化的RNN结构。
\begin{figure}[H]
    \centering
    \includegraphics[width=0.6\textwidth]{figure/RNN-unrolled.png}
    \caption{RNN的一般表示和展开表示}
    \label{fig:RNN_unrolled}
\end{figure}

% RNN细分
根据是否使用了某种门单元,RNN可分为普通RNN\upcite{DBLP:conf/interspeech/MikolovKBCK10},
LSTM\upcite{DBLP:journals/neco/HochreiterS97}和GRU\upcite{DBLP:conf/emnlp/ChoMGBBSB14}。
根据是否对反向序列(Reversed Sequence)建模,RNN可分为单向RNN(Unidirectional RNN)
和双向RNN(Bidirectional RNN)\upcite{DBLP:journals/tsp/SchusterP97}。
由于普通RNN受到梯度消失的影响,目前学界普遍采用LSTM或者GRU;
尽管后者受到梯度爆炸的影响,但是可以通过梯度剪裁(Gradient~Clipping)
\upcite{DBLP:journals/corr/VinyalsL15}解决。
采用多层RNN组成的深度神经网络比单层RNN能获得更好的性能\upcite{DBLP:journals/corr/VinyalsL15}。

% RNNLM
RNN语言模型可以给出序列$X=x_1, x_2, \cdots, x_n$的概率分布:
\begin{align}
    p(X) = \prod_{i=1}^{n} p(x_i|x_1, x_2, \cdots, x_{i-1})
    \label{eqn:language_model_probability}
\end{align}

RNN语言模型\footnote{为了简洁起见,我们描述了普通RNN。LSTM和GRU有着更复杂的数学表达式。}
通过神经网络中的参数来估计公式~\ref{eqn:language_model_probability}~乘积中的一项:
\begin{align}
    p(x_i = w|x_1, x_2, \cdots, x_{i-1}) = \frac{\exp{o_{tw}}}{\sum_{v=1}^V \exp{o_{tv}}}
    \label{eqn:language_model_estimation}
\end{align}
$o_t$是RNN的在$t$时刻的输出向量,$V$是词汇表的大小。
公式~\ref{eqn:language_model_estimation}~的右边本质上是对一个长度为$V$的向量进行softmax运算。
$t$时刻的输出向量是由输出矩阵$W_{out}$和$t$时刻隐层状态$h_i$相乘得到的:
\begin{align}
    o_i &= h_i^T W_{out}
\end{align}
而$h_i$则是当前输入$x_i$和上一时刻的隐层状态$h_{i-1}$在输入矩阵$W_{in}$和循环矩阵$W_{hh}$分别作用后再相加的结果:
\begin{align}
    h_i &= \sigma \left( x_i^T W_{in} + h_{i-1}^T W_{hh} \right)
\end{align}
RNN语言模型在训练时最大化训练集上的句子的对数概率:
\begin{align}
    \mathit{L(X)} = \sum_{i=1}^n \log p(x_i|x_1, x_2, \cdots, x_{i-1})
\end{align}
在预测时,对模型输入消息$m$,从模型的给出的概率分布中用某种搜索方法,如Beam Search可得出响应$r$。

% Seq2Seq
Seq2Seq框架使用两个拥有独立参数的RNN分别作为编码器和解码器。
尽管编解码器不一定都使用RNN\upcite{DBLP:journals/corr/BahdanauCB14},本文仅关注使用RNN的Seq2Seq变体。
首先,编码器把输入序列$X$编码成一个定长向量$v$。
该向量又称为思考向量(Though Vector),是编码器完全读取输入序列后的隐层状态(Last Hidden State)。
接着,解码器以$v$为初始隐层状态(Initial Hidden State),像一个RNN语言模型一样对输出序列进行预测。
整个过程可以描述为:编码器把输入序列$X$变换成某种压缩编码$v$,再由解码器把$v$还原为另一个序列$Y$,
Seq2Seq把公式~\ref{eqn:generative_conditional_probability}~
作了如下转化:
\begin{align}
    p(y_1, y_2, \cdots, y_m|x_1, x_2, \cdots, x_n) = \prod_{i=1}^m p(y_i|v, y_1, y_2, \cdots, y_{i-1})
\end{align}
其中$v$是以$f$为门单元的编码器的最后一个隐层状态:
\begin{align}
    h_i &= f(x_i ,h_{i-1}) \\
    c &= h_n
\end{align}
编码器和解码器以同一个目标函数同时训练。
为了更好的处理长序列,Seq2Seq一般引入注意力机制(Attention Mechanism)
\upcite{
    DBLP:journals/corr/BahdanauCB14,
    DBLP:conf/emnlp/LuongPM15},使输入序列的信息不必全部通过固定长度的向量$v$传递。
解码器能自动关注和当前输出最相关的输入部分,实现输入序列与输出序列的对齐(Alignment)。
注意力机制使传统Seq2Seq模型对较长输入也具有鲁棒性。

% Beam Search,Greedy & Random Sample
生成式模型仅仅定义了条件概率$p(Y|X)$,在推理阶段(Inference),需要采用某种启发式搜索算法从概率分布中
生成输出$Y$。最简单的搜索算法是贪心搜索(Greedy Search):在每一时刻都输出条件概率最大的单词:
\begin{align}
    y_i = \argmax p(y_i|y_1, y_2, \cdots, y_{i-1}|X)
\end{align}
因为各个$y_i$的概率都不是独立的,而是受之前输出的单词的影响,贪心搜索不能保证得到概率最大的输出序列。

% Hierarchical

\section{自动化评价指标}\label{sec:automatic_metric}

\section{公开的对话数据集}\label{sec:public_dataset}

\section{本章小结}\label{sec:rw_conclusion}
%这一节总结所有提及的指标,给它们分一下类,然后简要说明每一类
%指标的特点,或者是它考察的方面。
