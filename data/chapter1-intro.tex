% !Mode:: "TeX:UTF-8"
% Author: Zhengxi Tian
% Email: zhengxi.tian@hotmail.com

\chapter{绪论}

\section{课题研究背景}
随着互联网技术和社交媒体的飞速发展,人们能够轻松地在各种社交平台上发布和分享自己的意见及评论~\upcite{DBLP:journals/access/QiuLZ18},
比如电影评论,商品评价,餐厅用餐评价等等。
人们也越来越习惯于在网上购买商品,选择电影等方面,从其他人中获取意见和信息。
因此,海量的情感态度意见和评论迅速聚集,这就需要有高效率的方法来提取并分析这些信息,以得到其具有的潜在价值,更好地了解客户的反馈和偏好。
情感分析也因此成为一个有效的方法用来分析这些在线的发表意见~\upcite{DBLP:journals/coling/ZhaoLX16}。

在情感分析领域,很多的研究学者提出了诸多有关情感分析的方法,这些方法主要分为两大类:
计算语言学方法和机器学习或深度学习方法~\upcite{DBLP:journals/pr/KimL14a}。
计算语言学方法是基于预先定义好的语义词典对单词进行情感打分。
传统的方法比如基于词典的情感分析~\upcite{DBLP:conf/wsdm/DingLY08},
语法规则和统计学方法~\upcite{DBLP:journals/widm/SchullerMV15}。
在这种情况下,大多数研究都需要依赖于词袋(bag-of-words)或bag-of-n-grams的方法,
从标记(tokenization),特征选择(features selection),句子分解(snetences parsing)和文本分类(document classification)等角度来分析文本。
例如,朴素贝叶斯(Naive Bayes),支持向量机(SVM),和研究者提出的其他分类器~\upcite{DBLP:conf/emnlp/PangLV02}被用来对电影评论进行情感分类。
以上这些方法虽然可以在一定程度上解决问题,但是他们也有较大的局限性,因为模型相对于来说较为浅层。
并且,在进行篇幅级别,语句更长的情感分析时,这些模型无法捕捉语句顺序的相关信息。
研究表明,语句顺序对于情感分类是非常有帮助的~\upcite{DBLP:conf/ijcnn/LanZLW16}。
因为人类在阅读文本的文本的时候,对于语法依赖性的的掌握是理解文本情感含义的基础。

\section{课题研究意义}
情感分析的基本定义是对新闻,电影评论,社会媒体评论等在线用户反馈,将带有情感态度的主观或
客观文本进行分析后,应用自然语言处理的方法进行处理,归纳和推理~\upcite{DBLP:journals/coling/WilsonWH09}。

在情感分析领域,研究者提出了诸多方法。
比如,早期提出的单标签学习(Single Label Learning: SLL)和多标签学习(Multi-label Learning: MLL)将基本的情感单位视为类标签,
然后计算标签分数以确定评论的情感态度。
此外,还有学者通过构建情感词典的方法,从计算语义学的角度来进行情感分析。
近年来,随着机器学习的发展,尤其是深度学习技术的飞速提升,大量的机器学习和深度学习方法在情感分析领域中得到了广泛的应用。
传统的方法也面临着泛化误差高和维数灾难的缺陷,
情感分析领域预测模型和框架具有实际应用的前提就是有效解决上述两个难题,同时保证高效的性能。

所以,本文主要通过深度学习的方法来研究情感态度的二分类问题,
即,判断一个评价文本是带有正向情感还是负向情感态度。
后续在四个标准数据集上的结果也表明,基于注意力机制的双向门控制循环单元的框架优于在深度学习领域现有的部分模型和框架。

\section{课题研究内容}
在深度学习领域,除了先进的网络架构的需要,为了更好地完成情感分类任务,还需要了解人们在
阅读信息文本时如何感知文本所传达的情感和态度。
人类在阅读文本时,对自然语言的感知能力是构建理解体系的一个重要基础,他们不仅需要有选择性地专注于情感词汇,
以获得文本所表达的特定信息,还要结合注意不同的信息流部分,建立一个总体的情感理解架构~\upcite{}~\upcite{DBLP:journals/fcsc/RongPOLX15}。
受这一现象的启发,本文旨在从“句子序列建模”和“词特征理解”两个角度来分析情感,对网络在线评论本文进行情感极性的分类。
“句子序列建模”是从语义依赖角度建立情感感知体系,而“词特征理解”则是从“词元级别”将句子细化,对其关键信息进行捕捉。

考虑到句子序列是语法依赖性理解的基础,本文首先对句子的结构进行建模,以了解单词和单词之间的潜在联系。
循环神经网络(Recurrent Neural Network: RNN)已经证明了相较于传统模型支持向量机~\upcite{DBLP:conf/naacl/KudoM01}和条件随机场(Conditional Random Fields)~\upcite{DBLP:conf/icml/LaffertyMP01},
它具备的学习句子中单词之间潜在关系的能力~\upcite{DBLP:journals/candc/RazaA16}。
尽管如此,循环神经网络所带来的梯度消失和爆炸问题也局限了它的表现能力。
为了解决这一缺陷,很多更高级的神经网络例如长短时记忆网络(Long Short-Term Memory: LSTM)~\upcite{DBLP:journals/neco/HochreiterS97}和门控制循环单元(Gated Recurrent Unit: GRU)~\upcite{DBLP:conf/ssst/ChoMBB14}
开始被应用于自然语言处理,尤其是情感分析领域。
作为LSTM的简化版,GRU因为既能保持LSTM的优势,又具备更高的计算效率被更加广泛地使用。

因此,在本文的研究中,本文首先采用了一个预注意力的双向循环GRU。
它能结合单个单词与其前后单词的信息,并且从句子两端出发,交互性地考虑语义依赖的潜在联系,
从而在情感识别中构建初步的理解体系。

当输入序列逐渐变得更长时,往往对于LSTM和GRU来说很难捕捉全局情感的重要的成分~\upcite{DBLP:conf/emnlp/WangHZZ16}。
但是,句子中特定的单词同样在情感分析中起到了至关重要的角色。
注意力是一种可以在较长句子序列中捕捉其关键信息的机制~\upcite{DBLP:journals/corr/BahdanauCB14},
它有助于从“词元级别”对情感进行分类。

不仅如此,本文还模仿解码器的功能,在框架的底部搭建了一个后注意力单向GRU。
其主要作用在于,基于预注意力双向循环GRU和注意力机制两层架构,
提取对于情感预测的特征,最后输入到softmax分类器中得到最终情感预测的结果。
因此,本文的框架的优势和创新在于,可以从“句子序列建模”和“词特征捕捉”两个角度对文本情感极性进行预测和分类。

\section{论文组织结构}
本论文主要研究了深度学习,尤其是注意力机制和循环神经网络在情感分析中的应用。
考虑到情感分析领域中传统模型的浅层缺陷,和现在被广泛使用的神经网络的优缺点和性质。
本文提出了一个基于注意力机制的双向循环门控制单元框架,并且在四个标准数据集上进行了实验和量化分析。

本文的组织结构如下:

第一章:绪论。在绪论章节中,本文首先阐述了本课题的研究背景,并大致介绍了情感分析的研究现状和发展趋势等方面。
接下来,本文介绍了研究意义。其次,本文重点阐明了课题的研究内容和创新点。

第二章:相关知识和工作介绍。在做研究期间,本文的研究者查阅了大量的国内外相关文献,对情感分析中涉及到的知识做了详尽的说明。
包含计算语言学的普适方法极其局限性,机器学习和深度学习方法在情感分析中的相关应用。
并着重介绍了基于循环神经网络的语言模型,双向循环神经网络,门控制循环单元和注意力机制。

第三章:基于注意力机制的双向循环门控制单元框架。该章是本文研究的核心内容,也是本文的创新所在。
本章首先介绍了框架的详细结构,接下来,分别详细介绍了框架的主要三部分:“句子序列建模”的预注意力双向循环GRU,
“词特征捕捉”的注意力层设计,类解码器的后注意力单向GRU层。

第四章:实验结果与讨论。本章主要介绍了框架在四个标准数据集上的实验效果。首先介绍了公开的标准数据集极其组成。
然后详细阐述了评价方法,模型训练过程,超参数调节和实验结果等。最后从各个角度,例如对词嵌入进行可视化等,全面对实验进行了分析和讨论。

第五章:总结与展望。该章节对论文进行了总结,并指出工作中未来的研究方向。
