% !Mode:: "TeX:UTF-8"
% Author: Zhengxi Tian
% Email: zhengxi.tian@hotmail.com

\chapter{相关工作}\label{ch:related_work}
\section{自动化指标的使用情况}
%
生成式对话系统最早在2011年开始被研究。
Ritter等人率先用统计机器翻译(Statistical Machine Translation,SMT)的方法\upcite{Ritter:2011:DRG:2145432.2145500}解决响应生成问题。他们没有考虑上下文,而是把一个消息直接翻译为响应。Sordoni等人首次把神经网络用于响应生成,用神经网络对SMT输出的响应进行重现排序\upcite{DBLP:journals/corr/SordoniGABJMNGD15}。他们的模型属于在机器翻译中非常有效的编码器-解码器架构(Encoder-Decoder),对输入的上下文和消息进行词袋化(Bag-of-Word)后,用一个前馈神经网络(Feed-Forward Neural Network)进行编码,再用一个朴素的\footnote{称其为朴素的是因为它没有使用LSTM或者GRU,仅用了$\sigma$激活函数。}(Recurrent Neural Network,RNN)构成的语言模型(Language Model,LM)\upcite{DBLP:conf/interspeech/MikolovKBCK10}进行解码,根据上下文和消息的特征组合方式不同,他们提出了三个模型:RLM,DCGM-1和DCGM-2。Sutskever等人于2014年提出了基于RNN的编解码器,也就是序列到序列架构(Seq2Seq)\upcite{DBLP:conf/nips/SutskeverVL14}。

%这一节讲哪些人的模型用了哪些指标去测评,以及它们对指标的看法是怎么样的。

\section{自动化指标的开发情况}
%这一节讲有哪些人提出了针对生成式对话的特点提出了新指标。
%以及哪些人作了指标方面的实证研究。

\section{本章小结}\label{sec:本章小结}
%这一节总结所有提及的指标,给它们分一下类,然后简要说明每一类
%指标的特点,或者是它考察的方面。
