% !Mode:: "TeX:UTF-8"
% Author: Zhengxi Tian
% Email: zhengxi.tian@hotmail.com

\chapter{研究方法}\label{ch:研究方法}

\section{一般实验方法}\label{sec:general_experiment_method}
我们的研究是实证研究,就是通过设计和进行实验,从实验数据中得出结论的研究方法。
一般的来说,我们的实验是研究数据驱动的系统在一系列评价指标中的表现。
深度学习的模型都是数据驱动的系统,因此我们的实验方法具有一定的通用性。
从数学上讲,实验的对象是模型、数据集和指标,分别用$m\in M, d\in D, s\in S$表示,
其中$M, D, S$分别是模型、数据集和指标的集合。
我们把在一个数据集$d$上训练的模型$m$称为一个系统,记为$(m, d)$,它是能根据特定输入产生输出并被指标所评价的独立个体。
把三元组$(m, d, s)$称为一组实验的参数, 也可以用三元组表示一组实验,记作$E_{m, d, s}$。
一组实验能够产生某种可以数值化的结果$\hat{r}$,它是某个复杂函数$\mathit{F}$的输出$r$的观测值。
\begin{align}
    r &= f_s\left( f_{m, d}(x) \right) \\
    \mathit{F} &= f_s \circ f_{m, d}
\end{align}
在上述公式中,$f_{m, d}$表示一个系统$(m, d)$,$x$是系统的输入;$f_s$表示一个指标对系统的输出进行评价,$r$被定义为
评价的得分。在实际场景中,$f_s$可能还有其他输入。

$\mathit{F}$的潜在的参数是实验的参数$(m, d, s)$中的某些分量,$\mathit{F}$的自变量所在的集合$X$与具体领域有关。
我们把$r$看做一个随机变量,通过研究它的性质来研究复杂函数$\mathit{F}$的性质,
实际上,我们通过$r$的观测值$\hat{r}$来研究$r$的分布。
$\mathit{F}$的潜在参数中包含了指标,这在分类问题中可能是不必要的,因为分类系统的输出非常容易解释,举个例子:
图像分类系统的输出可以解释为:有多大概率这张图片是一只猫?
但是有些系统的输出并不容易解释或者评价,比如对话生成系统的输出就无法用“准确率”来衡量。
对这些系统来说,评价指标也是实验的重要参数。

我们对$\hat{r}$的研究主要采用控制变量法,即控制$(m, d, s)$中的一个或两个不变,观察$\hat{r}$的变化情况。
这种方法本质上通过控制输入$x$不变,$\mathit{F}$的其他参数不变,来观察某个参数$\lambda$对$\mathit{F}$的影响,
换句话说,$\mathit{F}$是如何作用于$\lambda$的。
这种方法可能是粗粒度的,因为它无法直接研究$r$的统计性质。
但是,当我们还不清楚$r$的统计性质和$\mathit{F}$的特性有何种关系时,控制变量法提供了很好的可解释性,举个例子,
我们固定一个指标$s$,对比两个模型$m_1, m_2$在一系列数据集$D$上的表现,如果$r_{m_1, d, s} > r_{m_2, d, s}$
对任意$d\in D$都成立,我们就可以从经验上说,在一系列数据集上,$m_1$的性能优于$m_2$,
尽管我们并不知道$m_1, m_2, s, d$各自对$\mathit{F}$的贡献是什么。


\section{生成式对话系统的实验方法论}
上一节为我们提供了研究一般的数据驱动系统在不同指标上的表现的方法,现在把它应用到生成式对话系统上来。
我们研究的模型是以Seq2Seq模型为框架的对话生成模型;
数据集是公开的对话数据集,一个数据集由多个对话实例(Dialogue Example)组成,每个对话实例
是一个二元组$(context, reference)$,$context$是对话的上下文,可由多个句子组成,
$reference$是对话的参考响应,是对上下文的合理答复,也可以由多个句子组成。
模型被训练为能从$context$生成合理的$response$;$context$是模型的输入, $response$是模型的输出。


与之前的工作不同,我们主要研究的是句子水平的得分,而不是系统水平的得分。
所谓系统水平的得分,就是一个系统在一个测试集上得出的总分
。BLEU就是作为一个系统水平的评价指标而提出的,许多对话研究者也都采用系统水平的得分。
然而,在我们的研究中,如果采用系统水平得分,就会受到数据稀疏问题的影响,因为一个$(m,d,s)$三元组只能
产生一个数字,这样我们全部可得的数据就取决于三元组的数量,这是非常少的。另一方面,系统水平的得分具有粗粒度性,
它最常见的用途就是比较两个系统的综合表现;而我们想要捕捉不同模型、指标和数据集互相组合所产生的细粒度信息,
例如,当模型、指标都相同,只有数据集不同,这种变化可能导致系统水平得分的变化是细微的,这样就不能反映变化所带来的影响。

句子水平的得分和系统的得分都是可以获得的,对于那些只提供系统水平得分的指标,如BLEU,我们把每一个句子看做一个只有
单个句子的文库即可;对于那些值提供句子水平得分的指标,我们把句子水平得分的算术平均值作为系统水平的得分。
这样,我们就能够同时从细粒度的句子水平得分和粗粒度的系统水平得分去考察一个系统。

\section{指标测评框架}\label{sec:指标测评框架}

\section{模型说明}\label{sec:模型说明}
这一节介绍实验涉及的模型。

\section{数据集说明}\label{sec:数据集说明}
这一节介绍实验涉及的数据集。

\section{指标说明}\label{sec:指标说明}
这一节介绍实验的指标。

\section{本章小结}\label{sec:本章小结2}
这一节为我们的方法找一些数学上的理论支持。
从数学上说明这些数据有什么意义。
